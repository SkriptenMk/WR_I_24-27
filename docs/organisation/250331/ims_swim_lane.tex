\documentclass{standalone}
\usepackage{tikz}
\usetikzlibrary{%
                backgrounds,
                positioning, 
                calc, 
                decorations.pathreplacing,
                }

\begin{document}
\tikzset{
    bahn/.style={
        draw,
        rectangle,
        anchor=center,
        transform shape,
        rotate=90,
        minimum width=1.2cm,
        text centered
        },
    wochentag/.style={
        draw,
        rectangle,
        minimum width=5cm,
        text centered,
        anchor=south west,
        },
        fach/.style={
        draw,
        rectangle,
        minimum size=5mm,
        },
    lesson-connection/.style={
        thick,
        rounded corners=2mm,
        to path={
            let \p1 = ($(\tikztotarget.center)-(\tikztostart.center)$),
                \n1 = {veclen(\x1,\y1)} in
            -- ++(0.4,0)                  % Kleine Strecke nach rechts
            -- ++(0,\y1/2)               % Halbe Vertikaldistanz hoch
            -- ++(-0.8,0)                  % Doppelte Strecke nach links
            -- ++(0,\y1/2)               % Restliche Vertikaldistanz
            -- ++(.4,0)                  % Zurück nach rechts
            -- (\tikztotarget)           % Zum Zielpunkt
            },
          ->,
        },
    lesson-connection2/.style={
        rounded corners=2mm,
        to path={
            let \p1 = ($(\tikztotarget.center)-(\tikztostart.center)$),
                \n1 = {veclen(\x1,\y1)} in
            -- ++(0.5,0)                  % Kleine Strecke nach rechts
            -- ++(0,0.75*\y1)            % 3/4 der Vertikaldistanz
            -- ++(-1,0)                   % Doppelte Strecke nach links
            -- ++(0,0.25*\y1)            % 1/4 der Vertikaldistanz
            -- ++(.5,0)                   % Zurück nach rechts
            -- (\tikztotarget)            % Zum Zielpunkt
            },
            },
    lesson-connection3/.style={
        rounded corners=2mm,
        to path={
            let \p1 = ($(\tikztotarget.center)-(\tikztostart.center)$),
                \n1 = {veclen(\x1,\y1)} in
            -- ++(0.5,0)                  % Kleine Strecke nach rechts
            -- ++(0,0.25*\y1)            % 3/4 der Vertikaldistanz
            -- ++(-1,0)                   % Doppelte Strecke nach links
            -- ++(0,0.75*\y1)            % 1/4 der Vertikaldistanz
            -- ++(.5,0)                   % Zurück nach rechts
            -- (\tikztotarget)            % Zum Zielpunkt
            },
        }
    }
\begin{tikzpicture}%[show background grid]
    \draw[dashed] (-.75,-.75) -- (30.25,-.75);
    \draw[dashed] (-.75,.75) -- (30.25,.75);
    \draw[dashed] (-.75,2.25) -- (30.25,2.25);
    \draw[dashed] (-.75,3.75) -- (30.25,3.75);
    \draw[dashed] (-.75,5.25) -- (30.25,5.25);
    \draw[dashed] (-.75,6.75) -- (30.25,6.75);
    \draw[dashed] (-.75,8.25) -- (30.25,8.25);
    \draw[dashed] (-.75,9.75) -- (30.25,9.75);
    \draw[dashed] (-.75,11.25) -- (30.25,11.25);

    \node[bahn] (S)   at (0,0)    {Sport};
    \node[bahn] (GuP) at (0,1.5)  {GuP};
    \node[bahn] (WR)  at (0,3)    {WR};
    \node[bahn] (FRW) at (0,4.5)  {FRW};
    \node[bahn] (M)   at (0,6)    {M};
    \node[bahn] (E)   at (0,7.5)  {E};
    \node[bahn] (F)   at (0,9)    {F};
    \node[bahn] (D)   at (0,10.5) {D};

    \node[wochentag] (Mo) at (1, 11.5)  {Montag};
    \node[wochentag] (Di) at (7, 11.5)  {Dienstag};
    \node[wochentag] (Mi) at (13, 11.5) {Mittwoch};
    \node[wochentag] (Do) at (19, 11.5) {Donnerstag};
    \node[wochentag] (Fr) at (25, 11.5) {Freitag};

    \node[fach, right= of WR.center] (a) {};
    \node[fach, right= 0mm of a] (b) {};
    \node[fach, right= 2cm of M.center] (c) {};
    \node[fach, right= 0mm of c] (d) {};
    \node[fach, right= 3cm of D.center] (e) {};

    \node[fach, right= 7cm of FRW.center] (f) {};
    \node[fach, right= 0mm of f] (g) {};
    \node[fach, right= 8cm of F.center] (h) {};
    \node[fach, right= 8.5cm of E.center] (i) {};
    \node[fach, right= 9cm of GuP.center] (j) {};
    \node[fach, right= 0mm of j] (k) {};
    \node[fach, right= 10cm of D.center] (l) {};
    \node[fach, right= 0mm of l] (m) {};

    \node[fach, right= 19cm of WR.center] (n) {};
    \node[fach, right= 19.5cm of FRW.center] (o) {};
    \node[fach, right= 20cm of F.center] (p) {};
    \node[fach, right= 0mm of p] (q) {};
    \node[fach, right= 21cm of M.center] (r) {};
   
    \node[fach, right= 25cm of E.center] (s) {};
    \node[fach, right= 0mm of s] (t) {};
    \node[fach, right= 26cm of S.center] (u) {};
    \node[fach, right= 0mm of u] (v) {};

    \draw[->] (b) to[lesson-connection] (c);
    \draw[->] (d) to[lesson-connection] (e);
    \draw[rounded corners=2mm, ->] (e) -- ++ (0.5,0) |- (f);
    \draw[->] (g) to[lesson-connection] (h);
    \draw[->] (h) to[lesson-connection] (i);
    \draw[->] (i) to[lesson-connection] (j);
    \draw[->] (k) to[lesson-connection2] (l);
    \draw[rounded corners=2mm, ->] (m) -- ++ (0.5,0) |- (n);
    \draw[->] (n) to[lesson-connection] (o);
    \draw[->] (o) to[lesson-connection3] (p);
    \draw[->] (q) to[lesson-connection] (r);
    \draw[rounded corners=2mm, ->] (r) -- ++ (0.5,0) |- (s);
    \draw[->] (t) to[lesson-connection] (u);


\end{tikzpicture}
\end{document}
