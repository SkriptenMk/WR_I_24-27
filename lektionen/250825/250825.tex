\documentclass[a4paper]{scrreprt}
\usepackage[german]{babel}
\usepackage[utf8]{inputenc}
\usepackage{array}
\usepackage{booktabs}
\usepackage{longtable}
\usepackage{ragged2e}
\usepackage{enumitem}
\setlist[itemize]{noitemsep}

\begin{document}
\section*{25.\ August 25, 2I, Produktlebenszyklus}
\begin{longtable}{p{1.5cm}>{\RaggedRight}p{7.5cm}p{2.5cm}}
    \toprule
    \emph{Zeit}&\emph{Inhalt}&\emph{Methode}\\
    \midrule
    \endhead

    \midrule
    \multicolumn{3}{c}{\begin{tiny}\textit{to be continued}\end{tiny}}\\
    \midrule
    \endfoot

    \bottomrule
    \endlastfoot

    1515&Wer erinnert sich an den iPod?&Lehrgespräch\\
        &Wer hätte zur Markteinführung einen iPod gekauft?&\\
    1520&Kein Gewinn zu Beginn der Verkaufsphase eines neuen
    Produktes&\\
        &Tesla (Markteinführung 2008, Gewinn erstmals 2020)&\\ [5pt]

    1525&Beispiel für ein Produkt in der Reifephase:
    Coca-Cola&Lehrgespräch\\ [5pt]

    1530&Klassische Smartphones als Beispiel für ein Produkt in der
    Sättigungsphase&Lehrgespräch\\
        &Abgrenzung zwischen einem einzelnen Produkt und einer
        Produktkategorie&\\ [5pt]

    1535&DVD als Produkt in der Degenerationsphase&Lehrgespräch\\ [5pt]

    1540&Relaunch und Produktinnovation&Partnerarbeit\\
        &Suchen Sie Beispiele für Relaunches bzw. Produktinnovationen&\\
        &Nokia 3310 als Beispiel&\\
    1550&Besprechung der Resultate&\\

\end{longtable}
\end{document}
