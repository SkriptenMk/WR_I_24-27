\documentclass[a4paper]{scrreprt}
\usepackage[german]{babel}
\usepackage[utf8]{inputenc}
\usepackage{array}
\usepackage{booktabs}
\usepackage{longtable}
\usepackage{ragged2e}
\usepackage{enumitem}
\setlist[itemize]{noitemsep}

\begin{document}
\section*{1I, 7. April 25, Projektmanagement (Zeitplanung \& Scrum)}
\begin{longtable}{p{1.5cm}>{\RaggedRight}p{7.5cm}p{2.5cm}}
    \toprule
    \emph{Zeit}&\emph{Inhalt}&\emph{Methode}\\
    \midrule
    \endhead

    \midrule
    \multicolumn{3}{c}{\begin{tiny}\textit{to be continued}\end{tiny}}\\
    \midrule
    \endfoot

    \bottomrule
    \endlastfoot

    1040&Rückblick auf die Prozessdarstellung&Vortrag\\ [5pt]

    1045&Grundsätze der Zeitplanung&Lehrgespräch\\
        &Verweis auf die Bedeutung der eigenen Agenda&\\ [5pt]

    1050&Traditionelles Projektmanagement&Vortrag\\
        &(Wasserfall: Analyse, Planung, Durchführung, Evaluation)&\\
        [5pt]

    1055&Kritische Würdigung Wasserfall&Lehrgespräch\\
        &\textit{Pro}: klare Verantwortlichkeiten, im Garantiefall klare
        rechtliche Spielregeln (Werkvertrag)&\\
        &\textit{Con}: Silodenken, unflexibel, grosse Projekte bieten zu
        viele unbekannte (Kontrollillusion)&\\

    1100&Agiles Manifest&Partnergespräch\\
        &Lesen, Besprechen der Aussage&\\
        &Zusammentragen der Resultate&Lehrgespräch\\
        &\textit{Hinweis}: rechts ist nicht unwichtig&\\ [5pt]

    1105&Einführung Scrum&Vortrag\\
        &\begin{itemize}
            \item[--] Aufteilen des Projekts in Userstorys $\rightarrow$
                Backlog
            \item[--] Sprint als Projektschritt
            \item[--] Sprintplanning $\rightarrow$ Sprintbacklog
            \item[--] Daily Scrum (Standup):
                \begin{itemize}
                    \item[--] Was habe ich gemacht
                    \item[--] Was mache ich heute
                    \item[--] Wo brauche ich unterstützung
                \end{itemize}
            \item[--] Sprintreview
        \end{itemize}
        Am Ende jedes Sprints soll das Projekt funktionieren und ein
        Schritt weiter sein (Inkrement).&\\ [5pt]

      1110&Anwendungsübung&Gruppenarbeit\\
          &Die Schüler erstellen gemeinsam eine illustrierte
          Zusammenfassung der bisherigen Inhalte zum Thema Organisation
          und Prozessmanagement. Sie verwenden dabei die durch GitHub
          zur Verfügung gestellten Hilfsmittel.&\\

\end{longtable}
\end{document}
